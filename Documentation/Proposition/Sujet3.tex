\documentclass[a4paper,12pt]{article}

\usepackage[french]{babel}
\usepackage[utf8]{inputenc}
\usepackage[T1]{fontenc}

\title{Sujet 3}
\author{HAMED Rémy, SOARES Enzo, TURBIEZ Denis}
\date{20 Mars 2022}

\begin{document}

\maketitle
\newpage

\tableofcontents

\newpage
\part{Proposition de Sujet}

\paragraph{Fourniture de repas pour personne en situation de dépendance}
Avec l'âge les personnes agée ne sont plus forcément capable de se faire manger et de s'occuper d'elles-même. Notre service propose de la distribution de repas et la mise en relation avec des personnes pour rendre ce moment plus convivial afin de casser la solitude et la monotonie qu'une personne agée seul chez elle peut rencontrer

\newpage
\part{Descriptif fonctionnel}

\section{Client}
Le client n'intéragie pas forcément avec l'application, une fois par semaine un operateur passe chez le/la client(e) et organise les menus de la semaines et chois les accompagnateurs pour les repas.

\section{Prestataire}
Les accompagnateurs devront être validé par la plateforme s'ils se présentent individuellement. Mais cette validations peut être délégué a une association qui aurait elle même au préalable validé l'accompagnateur.

l'entreprise gérer elle même les menu. L'application permet la gestion des stockes des danrées et des livraisons

\newpage
\part{Traitement Client}

\section{Client}
Il devra pouvoir chercher selectionner un repa, noter le repas.\\
Il devra pouvoir validé la préstation de l'accompagnateur.\\

\section{Prestataire}
 l'accompagnant devra avoir accés au informations sur la personne qu'il l'accompagne.\\
 il devra aussi pouvoir refuser ou accepter une mission.\\

Le operateurs des centres de distributions, doivent pouvoir préparer les commandes.\\
Valider la bonne doistribution des repas.\\

\newpage
\part{Référent Visuel}

\end{document}