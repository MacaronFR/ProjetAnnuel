\documentclass[a4paper,12pt]{article}

\usepackage[french]{babel}
\usepackage[utf8]{inputenc}
\usepackage[T1]{fontenc}
\usepackage{graphicx}

\title{Sujet 1}
\author{HAMED Rémy, SOARES Enzo, TURBIEZ Denis}
\date{20 Mars 2022}

\begin{document}

\maketitle
\newpage

\tableofcontents

\newpage
\part{Proposition de Sujet}

\paragraph{Accompagnement de personne présentant une invalidité temporaire dans leur déplacement hors de leur domicile}
En cas d'invalidité temporaire, suite à un accident, une opération …, proposer une plateforme pour que ces personnes puissent se rendre à d'autre endroits en utilisant un moyen de locomotion (transport en commun, voiture personnelle, voiture du prestataire …) à l'aide d'un prestataire afin de facilité le déplacement.

\newpage
\part{Descriptif fonctionnel}

\section{Client}
En utilisant un service de localisation et de trajet, le client va sélectionner un horaire, un point de départ ainsi qu'une destination. Les prestataires qui répondront a ce trajet seront afficher sur une liste pour le client et il pourra choisir le prestataire lui convenant le mieux en fonction de différents critère (services proposé, prix, proximité, type de véhicule).
Le client une fois le trajet terminé pourra alors donner son avis sur le prestataire ainsi qu'une note.

\section{Prestataire}
En utilisant un service de localisation, le prestataire pourra voir les demande de trajet dans un rayon défini par lui et calculer possible (pas d'offre commençant dans 10 minutes à plus de 10 minutes de trajet de la position actuelle), il pourra alors postuler à l'offre la plus avantageuse pour lui en proposant un prix. Quand le client le sélectionnera pour un trajet, le prestataire devra se rendre à l'adresse de départ afin de prendre en charge la personne en question, puis l'amener jusqu'à la destination.

\newpage
\part{Traitement Client}

L'application devra offrir les même possibilité sur un matériel mobile ou un matériel de bureau en privilégiant l'ergonomie mobile 

\section{Client}
Le client devra pouvoir exécuter une certaine quantité d'action avec l'application.\\
Il devra pouvoir chercher un itinéraire, avec un horaire de départ, et valider cet itinéraire afin qu'un prestataire puisse le prendre en charge.\\
Il devra pouvoir visualiser la liste des prestataire ayant répondu à sa demande.\\
Il devra pouvoir sélectionner le prestataire correspondant le mieux à ses attentes après avoir pu voir le détail de chacun.\\

\section{Prestataire}
Le prestataire devra pouvoir exécuter une certaine quantité d'action avec l'application afin de remplir son rôle au mieux.\\
Ainsi il devra pouvoir visualiser une liste de demande en fonction de certain paramètre dont notamment sa localisation.\\
Il devra pouvoir se positionner sur une demande de déplacement et être notifié en cas d'acceptation de la part du client.\\
L'application devra fournir un support pour les trajets afin de facilité et de centralisé les actions du prestataire.\\
Il devra pouvoir présenté un détails des services qu'il peut offrir dans le cadre de l'assistance de personne temporairement invalide aux déplacements.

\newpage
\part{Référent Visuel}

\section{Logo}
\includegraphics{sujet1.png}

\section{Nom de l'entreprise}
Le nom de l'entreprise doit évoqué le déplacement peut importe le moyen comme le rappel le logo.\\
C'est pourquoi nous avons choisi comme nom pour notre service \textbf{Metropolis}

\section{Slogan}
L'important n'est pas la destination

\end{document}